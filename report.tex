\documentclass[12pt, oneside]{article}
\usepackage[margin=1in]{geometry}
\geometry{letterpaper}
%\geometry{landscape}
%\usepackage[parfill]{parskip}
\usepackage{graphicx}
\usepackage{amsmath}
\pagenumbering{arabic}		
\usepackage{amssymb}
\usepackage{pdfpages}
\usepackage{listings}
\usepackage{mathtools}
\usepackage[utf8]{inputenc}


\title{STAT 243 Final Project\\Implementation of an Adaptive-Rejection Sampling Method}
\author{Ye Zhi, Boying Gong, Max Gardner, Alexander Brandt}
\date{December 17, 2015}

\DeclarePairedDelimiter\ceil{\lceil}{\rceil}
\DeclarePairedDelimiter\floor{\lfloor}{\rfloor}

\newcommand{\cpder}[3]{\left(\frac{\partial#1}{\partial#2}\right)_{#3}} %Constrained partial derivate
\newcommand{\pder}[2]{\left(\frac{\partial#1}{\partial#2}\right)} %Partial derivate
\newcommand{\dbar}{\ensuremath{\mathchar'26\mkern-12mu d}}
\newcommand{\nPr}[2]{ {_{#1}}P_{#2} } %Partial derivate

\begin{document}

\maketitle

\section{Theory}

Our final project was to implement an adaptive-rejection sampling method in R.  Our method largely implements the methods described by Gilks and Wild.  ARS attempts to efficiently generate random values according to a user-defined distribution (and a user-defined domain).  The method follows a few general steps, exploiting the tendency of many important and useful distributions to be log-concave (even if the distributions themselves are not concave).  For some distribution \(g(x)\), let us define \(h(x) = \log(g(x))\).  Note, this point, if \(h(x)\) is found to be not log-concave, then our program will reject the values.
\\\\
Now, in order to efficiently sample from values (and not waste time generating values that will ultimately be rejected), we construct an upper envelope function that bounds \(h(x)\).  This is performed by taking two or more starting points from within the domain such that they lie within our users defined bounds.  At points we will calculate the tangent's slope at \(x_i\).  We then calculate the x-intercept where these slopes meet with:
\[ z_j = \frac{h(x_{j+1}) - h(x_j) - x_{j+1}h'(x_{j+1}) + x_j h'(x_j)}{h'(x_j) - h'(x_{j+1})} \]
Where \(j \in [1,k-1]\).  Note that \(z_0\) and \(z_k\) are the upper and lower bounds of the function, respectively, even if those bounds are \(-\infty\) or \(\infty\).
\\\\
Now that we can define the envelope for \(x \in [z_{j-1}, z_j]\):
\[ u_k(x) = h(x_j) + (x - x_j)h'(x_j) \]
By exponentiating and normalizing this \(u_k(x)\), we now have a density from which to sample our random values:
\[ s_k(x) = \frac{\exp(u_k(x))}{N} \]
Where N is simply the all-space integral value of \(s_k(x)\) (i.e. \(N = \int_D s_k(x) dx \)).

Furthermore, we exploit a computationally inexpensive procedure to compute a lower hull, which is defined as:

\[ l_k(x) = \frac{(x_{j+1} - x)h(x_j) + (x - x_j) h(x_{j+1})}{x_{j+1} - x_{j}} \]

\(l_k(x) \) can allow us to perform a squeezing test to quickly accept or reject proposed sampling values.  This is performed by selecting some value \(x^t\) from \(s_k(x)\), and drawing a value \(w\) from the uniform distribution defined from [0,1].  We can accept \(x^t\) if:

\[ w \le \exp(l_k(x^t) - u_k(x^t)) \]

Or if:

\[ w \le \exp(h(x^t) - u_k(x^t)) \]

These are the squeezing test and rejection test, respectively.  As we continue to compute \(h(x^t\) and \(h'(x^t)\), we can update the points that construct our upper and lower hulls, thereby increasing the chance of new values drawn from \(s_k(x)\) being accepted.  This is the mathematical and statistical intuition behind the implementation discussed in the next section.

\section{Implementation}

\section{Results/Examples}

We provide some basic results in addition to the extensive testing of each function individually, which can be seen in more detail in the test suite defined in the package.

\subsection{Normal Distribution}

\subsection{Chi-Squared Distribution}

\subsection{Truncated Normal Distribution}

\section{Roll the Credits}

Ye Zhi \\
Boying Gong \\
Max Gardner \\
Alexander Brandt wrote an initial implementation of the ARS and \LaTeX'd up the final report.  He is a computational biologist interested in worms, humans, and what they have in common (and what they don't!).  \\

\section{Citations}



\end{document}